\documentclass{sigchi}

% Use this section to set the ACM copyright statement (e.g. for
% preprints).  Consult the conference website for the camera-ready
% copyright statement.

% Copyright
\CopyrightYear{2020}
%\setcopyright{acmcopyright}
\setcopyright{acmlicensed}
%\setcopyright{rightsretained}
%\setcopyright{usgov}
%\setcopyright{usgovmixed}
%\setcopyright{cagov}
%\setcopyright{cagovmixed}
% DOI
\doi{https://doi.org/10.1145/3313831.XXXXXXX}
% ISBN
\isbn{978-1-4503-6708-0/20/04}
%Conference
\conferenceinfo{CHI'20,}{April  25--30, 2020, Honolulu, HI, USA}
%Price
\acmPrice{\$15.00}

% Use this command to override the default ACM copyright statement
% (e.g. for preprints).  Consult the conference website for the
% camera-ready copyright statement.

%% HOW TO OVERRIDE THE DEFAULT COPYRIGHT STRIP --
%% Please note you need to make sure the copy for your specific
%% license is used here!
% \toappear{
% Permission to make digital or hard copies of all or part of this work
% for personal or classroom use is granted without fee provided that
% copies are not made or distributed for profit or commercial advantage
% and that copies bear this notice and the full citation on the first
% page. Copyrights for components of this work owned by others than ACM
% must be honored. Abstracting with credit is permitted. To copy
% otherwise, or republish, to post on servers or to redistribute to
% lists, requires prior specific permission and/or a fee. Request
% permissions from \href{mailto:Permissions@acm.org}{Permissions@acm.org}. \\
% \emph{CHI '16},  May 07--12, 2016, San Jose, CA, USA \\
% ACM xxx-x-xxxx-xxxx-x/xx/xx\ldots \$15.00 \\
% DOI: \url{http://dx.doi.org/xx.xxxx/xxxxxxx.xxxxxxx}
% }

% Arabic page numbers for submission.  Remove this line to eliminate
% page numbers for the camera ready copy
% \pagenumbering{arabic}

% Load basic packages
\usepackage{balance}       % to better equalize the last page
\usepackage{graphics}      % for EPS, load graphicx instead 
\usepackage[T1]{fontenc}   % for umlauts and other diaeresis
\usepackage{txfonts}
\usepackage{mathptmx}
\usepackage[pdflang={en-US},pdftex]{hyperref}
\usepackage{color}
\usepackage{booktabs}
\usepackage{textcomp}
\usepackage{graphicx}

% Some optional stuff you might like/need.
\usepackage{microtype}        % Improved Tracking and Kerning
% \usepackage[all]{hypcap}    % Fixes bug in hyperref caption linking
\usepackage{ccicons}          % Cite your images correctly!
% \usepackage[utf8]{inputenc} % for a UTF8 editor only

% If you want to use todo notes, marginpars etc. during creation of
% your draft document, you have to enable the "chi_draft" option for
% the document class. To do this, change the very first line to:
% "\documentclass[chi_draft]{sigchi}". You can then place todo notes
% by using the "\todo{...}"  command. Make sure to disable the draft
% option again before submitting your final document.
\usepackage{todonotes}

% Paper metadata (use plain text, for PDF inclusion and later
% re-using, if desired).  Use \emtpyauthor when submitting for review
% so you remain anonymous.
\def\plaintitle{CSCM69: Human-Centred Perspectives and Methods\\Coursework 2 - Work/Life Balence}
\def\plainauthor{Andy Gray}
\def\emptyauthor{}
\def\plainkeywords{Authors' choice; of terms; separated; by
  semicolons; include commas, within terms only; this section is required.}
\def\plaingeneralterms{Documentation, Standardization}

% llt: Define a global style for URLs, rather that the default one
\makeatletter
\def\url@leostyle{%
  \@ifundefined{selectfont}{
    \def\UrlFont{\sf}
  }{
    \def\UrlFont{\small\bf\ttfamily}
  }}
\makeatother
\urlstyle{leo}

% To make various LaTeX processors do the right thing with page size.
\def\pprw{8.5in}
\def\pprh{11in}
\special{papersize=\pprw,\pprh}
\setlength{\paperwidth}{\pprw}
\setlength{\paperheight}{\pprh}
\setlength{\pdfpagewidth}{\pprw}
\setlength{\pdfpageheight}{\pprh}

% Make sure hyperref comes last of your loaded packages, to give it a
% fighting chance of not being over-written, since its job is to
% redefine many LaTeX commands.
\definecolor{linkColor}{RGB}{6,125,233}
\hypersetup{%
  pdftitle={\plaintitle},
% Use \plainauthor for final version.
%  pdfauthor={\plainauthor},
  pdfauthor={\emptyauthor},
  pdfkeywords={\plainkeywords},
  pdfdisplaydoctitle=true, % For Accessibility
  bookmarksnumbered,
  pdfstartview={FitH},
  colorlinks,
  citecolor=black,
  filecolor=black,
  linkcolor=black,
  urlcolor=linkColor,
  breaklinks=true,
  hypertexnames=false}

% create a shortcut to typeset table headings
% \newcommand\tabhead[1]{\small\textbf{#1}}

% End of preamble. Here it comes the document.
\begin{document}

\title{\plaintitle}

\numberofauthors{1}
\author{%
  \alignauthor{??\\
    \affaddr{??}\\
    \affaddr{Swansea, Wales}\\
    \email{??@swansea.ac.uk}}
}
\maketitle

\begin{abstract}
	
	
\end{abstract}


% ACM Classfication

\begin{CCSXML}
<ccs2012>
<concept>
<concept_id>10003120.10003121</concept_id>
<concept_desc>Human-centered computing~Human computer interaction (HCI)</concept_desc>
<concept_significance>500</concept_significance>
</concept>
<concept>
<concept_id>10003120.10003121.10003125.10011752</concept_id>
<concept_desc>Human-centered computing~Haptic devices</concept_desc>
<concept_significance>300</concept_significance>
</concept>
<concept>
<concept_id>10003120.10003121.10003122.10003334</concept_id>
<concept_desc>Human-centered computing~User studies</concept_desc>
<concept_significance>100</concept_significance>
</concept>
</ccs2012>
\end{CCSXML}

\ccsdesc[500]{Human-centered computing~Human computer interaction (HCI)}
\ccsdesc[300]{Human-centered computing~Haptic devices}
\ccsdesc[100]{Human-centered computing~User studies}

% Author Keywords
\keywords{\plainkeywords}

% Print the classficiation codes
\printccsdesc
Please use the 2012 Classifiers and see this link to embed them in the text: \url{https://dl.acm.org/ccs/ccs_flat.cfm}



\section{Introduction}
	Harvard conducted a survey which asked professional people how many hours they worked a week, 94\% said they put in more than 50 hours or more. Out of these professionals, 50\% said they are working 65 or more hours \cite{harvard_review}. What is even more staggering is that this survey got done in 2009, a time where Blackberry mobile phones were all the rage and iPhones had only been on the market for around two years. This year was when the iPhone 3G was just about to hit stores and was way before the iPhone 4 and where the smartphone, as we currently know them, indeed took off and changed the way we interact with our mobile devices. As the Harvard survey also found out that 20-25 hours a week get spent monitoring their Blackberrys while outside of working hours \cite{harvard_review}. 
	
	
	These numbers show that a work-life balance has been an issue for some years. Especially when looking at statistics published in 2020, by the NY Post's Business Insider, that state 48\% of Americans consider themselves workaholics and the CNBC stating that 66\% of American works lacking a healthy work-life balance \cite{work-life_2020}. A staggering fact that we can relate to from experience is that 77\% of full-time works suffer from burnout from their current job \cite{work-life_2020}. Rescue Time analysed their users' data in 2019 and found that 40\% of people used their computers after 10 pm and 28\% of people start their workday before 8:30 am \cite{rescuetime_study}.  
	
	What we aim to do within this report is to identify some of the leading apps that get associated with work-life balance. Once these apps get identified, we aim to investigate these tools while critiquing their designs concerning their interactions with HCI. These apps include \textbf{[list apps here]}, we will then be interviewing users and finding out their views on these applications and how they have impacted their work-life balance.
	
	%[Overview of the assignment]

\section{Find out what key existing apps, services and devices are widely used in the domain of Work-Life Balance}
		\subsection{Mobile Phones (Device)}
		\subsection{Smart Watch (Device)}
		\subsection{Email}
		\subsection{App timer notifiers - think apple screen time.}

\section{Investigate these tools, critiquing their design}
		\subsection{Mobile Phones (Device)}
		\subsection{Smart Watch (Device)}
		\subsection{Email}
		\subsection{App timer notifiers - think apple screen time.}
	
	

%\begin{figure}
%	\begin{center}
%		\includegraphics[width=5cm]{instagram_example2.jpeg}
%		\caption{An Instragram user profile overview. This is @bigdataqueen public profile overview.}
%		\label{fig:instagram_overview}
%	\end{center}
%\end{figure}


\section{Design an interview with users}

%\begin{figure}
%	\begin{center}
%		\includegraphics[width=5cm]{instagram_example2.jpg}
%		\caption{A design concept of @bigdataqueen new public profile overview.}
%		\label{fig:instagram_new_overview}
%	\end{center}
%\end{figure}

	


\section{Write up your results}

\section{Conclusion}
	 
	
	
% BALANCE COLUMNS
\balance{}

% REFERENCES FORMAT
% References must be the same font size as other body text.
\bibliographystyle{SIGCHI-Reference-Format}
\bibliography{sample}

\end{document}

%%% Local Variables:
%%% mode: latex
%%% TeX-master: t
%%% End:
